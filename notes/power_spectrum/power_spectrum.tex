\documentclass[a4paper,twosides]{article}
\usepackage[utf8]{inputenc}
\usepackage{amsmath}

\newcommand{\pypycontact}{\texttt{PyPyContact}}
\renewcommand{\d}{\ensuremath{\mathrm{d}}}
\newcommand{\e}{\ensuremath{\mathrm{e}}}
\renewcommand{\i}{\ensuremath{\mathrm{i}}}

\renewcommand{\vec}[1]{\ensuremath{\boldsymbol{#1}}}
\newcommand{\abs}[1]{\ensuremath{\left|#1\right|}}
\newcommand{\hrms}[1]{\ensuremath{h_{\mathrm{rms}}}}
\usepackage{natbib}

\usepackage[squaren]{SIunits}


\title{Power Spectrum Conventions for \pypycontact}
\author{Till Junge\\\texttt{till.junge@altermail.ch}}

\begin{document}
\maketitle
\begin{abstract}
  This document defines the convention used for the handling of power spectra within \pypycontact. The conventions are based on the draft ``Surface roughness and power spectra metrology'' by Tevis Jacob and Lars Pastewka.
\end{abstract}

\section{Definition of Fourier Transforms}
\label{sec:fourier_transforms}


\begin{equation}
  H(\vec q) = \mathcal{F}(h(\vec x)) = \int_Ah(\vec x)\e^{-\i\vec q\cdot\vec x}\d A,
  \label{eq:four}
\end{equation}
and has dimension of $[\metre^3]$.
\begin{equation}
  h(\vec x) = \mathcal{F}^{-1}(H(\vec q)) = \frac{1}{A} \sum_{\vec q_i} H(\vec q_i)\e^{\i\vec q_i\cdot\vec x}
 \label{eq:ifour}
\end{equation}
in $[\metre]$.

With these definitions, we can derive their discretised versions for the use in \pypycontact. The Fourier transform becomes
\begin{equation}
  H(\vec q_i) = \mathtt{FFT}(h(\vec x_j)) = \frac{A}{N}\sum_{\vec x_j}h(\vec x_j)\e^{-\i\vec q_i\cdot\vec x_j},
  \label{eq:fft}
\end{equation}
and its inverse is
\begin{equation}
  \label{eq:ifft}
  h(\vec x_i) = \mathtt{FFT}^{-1}(H(\vec q_j))= \frac{1}{A}\sum_{\vec q_j}H(\vec q_j)\e^{\i\vec q_j\cdot\vec x_i}
\end{equation}

\section{Definition of Power Spectrum}
\label{sec:power_spectrum}

The continuous power spectrum $C(q)$ is defined as
\begin{equation}
  \label{eq:cont_pow_spec}
  C(q) = \frac{1}{A} \abs{H(\vec q)}^2
\end{equation}

\section{Parseval's theorem}
\label{sec:parseval}

for continuous surfaces is 
\begin{equation}
  \label{eq:cont_parseval}
  \int_A\abs{h(\vec x)}^2\d A = \frac{1}{A}\sum_{\vec q_i}\abs{H(\vec q_i)}^2
\end{equation}
its discretised version is
\begin{equation}
  \label{eq:parseval}
  \frac{A}{N}\sum_{\vec x_i}\abs{h(\vec x_i)}^2 = \frac{1}{A}\sum_{\vec q_i}\abs{H(\vec q_i)}^2
\end{equation}

\section{Sythesis of Surfaces}
\label{sec:synthesis}

Only the more or less trivial case of a surface with an exact power spectrum is detailed here. The power spectrum $C(q)$ scales with $q$ as follows \citep{Persson2005}
\begin{equation}
  \label{eq:pow_spec_scale}
  C(q) \sim q^{-2-2H}, \qquad \Rightarrow C(q) = \beta^2\,q^{-2-2H}.
\end{equation}
The definition of the power spectrum can be turned around:
\begin{equation}
  \label{eq:coeffs}
  H(\vec q) = \sqrt{A\, C(q)} \e^{\i\phi},
\end{equation}
where $\phi$ in a randomly distributed phase. In order to obtain a real surface, the following symmetry must hold,

\begin{equation}
  \label{eq:symmetry}
  H(\vec q) = H^*(-\vec q).
\end{equation}
The scaling factor $\beta$ can be chosen to fit the properties of the surface. On obvious choice is to fit the rms height \hrms.
\begin{equation}
  {\hrms{}}^2 = \frac1A\int_A\abs{\Delta h(\vec x)}^2\d A,
\end{equation}
where $\Delta h$ is the offset from the mean plane through the rough surface. By assuming an infinite surface size, we can use Parseval's theorem (for details, check the draft)
\begin{equation}
  \label{eq:h_rms_of_C}
  {\hrms{}}^2 = \frac1{2\pi}\int_{q_0}^\infty q\, C(q)\,\d q,
\end{equation}

and substitute the scaling \eqref{eq:pow_spec_scale} into \eqref{eq:h_rms_of_C}
\begin{align}
  \label{eq:beta}
  {\hrms{}}^2 &= \frac{\beta^2}{2\pi}\int_{q_0}^\infty q^{-1-2H}\,\d q \notag\\
  &=\frac{\beta^2}{4H\pi q_0^{2H}}\\
  &\Rightarrow \beta = \sqrt{4H{\hrms{}}^2\pi q_0^{2H}}
\end{align}
\bibliographystyle{elsarticle-harv}%plainnat}
\bibliography{../biblio}
\end{document}

%%% Local Variables: 
%%% ispell-local-dictionary: "british"
%%% mode: latex
%%% TeX-master: t
%%% End: 
